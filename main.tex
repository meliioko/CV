%% If you need to pass whatever options to xcolor
\PassOptionsToPackage{dvipsnames}{xcolor}

%% If you are using \orcid or academicons
%% icons, make sure you have the academicons
%% option here, and compile with XeLaTeX
%% or LuaLaTeX.
% \documentclass[10pt,a4paper,academicons]{altacv}

%% Use the "normalphoto" option if you want a normal photo instead of cropped to a circle
% \documentclass[10pt,a4paper,normalphoto]{altacv}

\documentclass[10pt,a4paper,ragged2e,withhyper]{altacv}

%% AltaCV uses the fontawesome5 and academicons fonts
%% and packages.
%% See http://texdoc.net/pkg/fontawesome5 and http://texdoc.net/pkg/academicons for full list of symbols. You MUST compile with XeLaTeX or LuaLaTeX if you want to use academicons.

% Change the page layout if you need to
\geometry{left=1.25cm,right=1.25cm,top=1.5cm,bottom=1.5cm,columnsep=1.2cm}

% The paracol package lets you typeset columns of text in parallel
\usepackage{paracol}

% Change the font if you want to, depending on whether
% you're using pdflatex or xelatex/lualatex
\ifxetexorluatex
  % If using xelatex or lualatex:
  \setmainfont{Roboto Slab}
  \setsansfont{Lato}
  \renewcommand{\familydefault}{\sfdefault}
\else
  % If using pdflatex:
  \usepackage[rm]{roboto}
  \usepackage[defaultsans]{lato}
  % \usepackage{sourcesanspro}
  \renewcommand{\familydefault}{\sfdefault}
\fi
  % commentaires longs
  \usepackage{comment}

% Change the colours if you want to
\definecolor{SlateGrey}{HTML}{2E2E2E}
\definecolor{LightGrey}{HTML}{666666}
\definecolor{DarkPastelRed}{HTML}{450808}
\definecolor{PastelRed}{HTML}{8F0D0D}
\definecolor{GoldenEarth}{HTML}{E7D192}
\colorlet{name}{black}
\colorlet{tagline}{PastelRed}
\colorlet{heading}{DarkPastelRed}
\colorlet{headingrule}{GoldenEarth}
\colorlet{subheading}{PastelRed}
\colorlet{accent}{PastelRed}
\colorlet{emphasis}{SlateGrey}
\colorlet{body}{LightGrey}

% Change some fonts, if necessary
\renewcommand{\namefont}{\Huge\rmfamily\bfseries}
\renewcommand{\personalinfofont}{\footnotesize}
\renewcommand{\cvsectionfont}{\LARGE\rmfamily\bfseries}
\renewcommand{\cvsubsectionfont}{\large\bfseries}


% Change the bullets for itemize and rating marker
% for \cvskill if you want to
\renewcommand{\itemmarker}{{\small\textbullet}}
\renewcommand{\ratingmarker}{\faCircle}

%% sample.bib contains your publications
\addbibresource{sample.bib}
\usepackage{comment}

\begin{document}
\name{Mathieu LATOURNERIE}
\tagline{Élève ingénieur spécialisé en IA}
%% You can add multiple photos on the left or right
%\photoR{2.8cm}{Globe_High}
% \photoL{2.5cm}{Yacht_High,Suitcase_High}

\personalinfo{%
  % Not all of these are required!
  \email{mathieu.latournerie@epita.fr}
  \phone{+33 6 62 77 76 64}
  \mailaddress{14, Rue des Bergers, 85340 Les Sables d'Olonne France}
  %\location{Location, COUNTRY}
  %\homepage{www.homepage.com}
  %\twitter{@twitterhandle}
  %\linkedin{your_id}
  %\github{your_id}
  %% You MUST add the academicons option to \documentclass, then compile with LuaLaTeX or XeLaTeX, if you want to use \orcid or other academicons commands.
  %% You can add your own arbtrary detail with
  %% \printinfo{symbol}{detail}[optional hyperlink prefix]
  % \printinfo{\faPaw}{Hey ho!}[https://example.com/]
  %% Or you can declare your own field with
  %% \NewInfoFiled{fieldname}{symbol}[optional hyperlink prefix] and use it:
  % \NewInfoField{gitlab}{\faGitlab}[https://gitlab.com/]
  % \gitlab{your_id}
}

\makecvheader
%% Depending on your tastes, you may want to make fonts of itemize environments slightly smaller
%\AtBeginEnvironment{itemize}{\small}

%% Set the left/right column width ratio to 6:4.
\columnratio{0.55}

% Start a 2-column paracol. Both the left and right columns will automatically
% break across pages if things get too long.
\begin{paracol}{2}
\cvsection{Formation}

\cvevent{Formation ingénieur}{EPITA}{2021-2024}{}Majeure Data-Science et intelligence artificielle

\divider

\cvevent{Formation ingénieur}{École Mines de Saint Etienne Cursus ISIMIN}{2018-2021}{}

\divider

\cvevent{Classe préparatoire scientifique}{Lycée Chateaubriand}{2016-2018}{}
Math Physique option informatique

\begin{comment}
    
\divider

\cvevent{Lycée filière scientifique}{Lycée Sainte Marie du port}{2013-2016}{}
Bac mention très bien \\
Option spécialité math
\end{comment}

\cvsection{Expérience professionnelle}

\cvevent{Projet parabole}{CNES}{octobre 2019 -- 1 an}{}
\begin{itemize}
\item Projet de réalisation en partenariat avec le CNES
\item Création d'un appareil de mesure de masse en absence de \\
gravité, testé en vol zéro g
\end{itemize}

\divider

\cvevent{Stage - Base de données}{PROGINOV}{janvier 2019 -- 1 mois}{}
\begin{itemize}
\item Développement informatique, relation client
\item Mise en place d'un outil pour la gestion du parc\\
informatique
\item Gestion de base de données SQL
\end{itemize}

\divider

\cvevent{Stage - développement informatique}{DXOMARK}{Septembre 2022 -- 6 mois}{}
\begin{itemize}
\item Développement informatique en python d'outils internes
\item Développement frontend en VueJs pour un client.
\item Travail collaboratif
\end{itemize}

\cvsection{Langues}

\cvskill{Français (Natif)}{5}
\divider
\cvskill{Anglais }{4}
\divider
\cvskill{Japonais}{2}

\switchcolumn

\cvsection{Expérience académique}

\begin{comment}
\cvevent{Projet TIGER}{}{février 2022 -- 2 mois}{}
\begin{itemize}
\item Développement d'un compilateur en C++
\item Gestion du travail en équipe
\item Travail de lecture de code fourni
\end{itemize}

\divider

\cvevent{Projet JWS}{}{janvier 2022 -- 1 mois}{}
\begin{itemize}
\item Rush d'apprentissage du language Java
\item Test des compétences sur un projet final : Un serveur de jeu Bomberman
\item Découverte du framework quarkus et de l'utilisation d'une API REST
\end{itemize}

\divider

\cvevent{Projet 42SH}{}{décembre 2021 -- 1 mois}{}
\begin{itemize}
\item Développement d'un shell en C
\item Gestion du travail en équipe 
\end{itemize}

\divider

\cvevent{Python Big data}{}{Février 2023 -- 1 mois}{}
\begin{itemize}
    \item Travail avec pandas et numpy pour analyser des données
    \item Analyse de données sur la blockchain du bitcoin
\end{itemize}
\divider


\cvevent{Projet Harmony}{}{Mai 2023 -- 3 mois}{}
\begin{itemize}
    \item Projet de Data engineering
    \item Utilisation de Hadoop, notament HDFS, Spark et Kafka
    \item Traitement des données de façon distribué
\end{itemize}

\divider
\end{comment}

\cvevent{Projet IREN}{}{Mars 2023 -- 2 mois}{}
\begin{itemize}
    \item Reconnaissance d'image avec Tensoflow
    \item Création d'un réseau de neurones de classification sur des images de bateaux en entier, concours sur la précision
\end{itemize}

\divider

\cvevent{Projets NLP}{}{Mars 2023 -- 6 mois}{}
\begin{itemize}
    \item Multiple projets de NLP allant du Naive Bayes classifier au fine tuning de transformers
    \item Analyse des prédictions faites par les modèles pour comprendre les comportements et erreurs
    \item Projets fais en pytorch
\end{itemize}

\divider
\cvevent{Apprentissage par renforcement}{}{September 2023 -- 2 mois}{}
\begin{itemize}
    \item Implémentation du papier de DeepMind sur le DQN
    \item Objectif: faire jouer un agent à atari breakout et maximiser le score
    \item Implémentation complète en pytorch et fine-tuning des paramètres
\end{itemize}
\divider
\cvevent{Hackaton Emerton}{}{décembre 2023 -- 2 semaines}{}
\begin{itemize}
  \item Concours de projet de génération d'image
  \item Finetuning de Stable diffusion
  \item Utilisation de méthode state of the art
\end{itemize}
\cvsection{Informatique}

\begin{itemize}
    \item \textbf{Langages :} \\
            C ; C++ ; Java ; Python ; Scala ; SQL ; Js ; Scala ;
    \item \textbf{ML Frameworks:}\\
            Tensorflow ; Pytorch ; Spark ;
    \item \textbf{Outils :} \\
            Git, Linux, Docker, Kafka, Azure
\end{itemize}

\cvsection{Loisirs}
\cvevent{Badminton en compétion}{}{2012 -- Aujourd'hui}{}
\begin{itemize}
    \item Participation à de nombreux tournois et compétitions par équipe, niveau national
    \item Bénévolat et organisation de compétition

\end{itemize}

\end{paracol}
\end{document}